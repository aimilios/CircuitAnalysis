\documentclass{article}
\usepackage{amsmath}
\usepackage[american]{circuitikz}

\title{RC Circuit Transient Analysis using Laplace Transform for both DC and AC Voltage Source }
\author{Aimilios's Circuits}

\begin{document}

\maketitle

\section{RC Circuit}
\begin{circuitikz}[american voltages]
      \draw (0,0)
      to[V=$u_s$,invert,v=$u_s(t)$] (0,3)                % The Voltage source
      node[label={[font=\footnotesize]above:$u_A(t)$}]{} % Node A Voltage 
      to[R=$R$, v=$u_R(t)$, i=$i_R(t)$] (6,3)            % The Resistor
      node[label={[font=\footnotesize]above:$u_B(t)$}]{} % Node B Voltage 
      to[C=$C$, v=$u_C(t)$, i=$i_C(t)$] (6,0)            % The Capacitor
      to[short] (0,0);
      \draw (3,0)  node[ground]  {};                     % Ground Node
      \draw[thin, <-, >=triangle 45] (3,1.25)node{$i(t)$}  ++(-60:0.5) arc (-60:170:0.5);  %Mesh Current Arrow
\end{circuitikz}

\subsection{V-I Relations}  % V-I Relations
$u_R(t)=R\cdot i_R(t)$   \hspace{3.1cm}  $i_R(t)=\frac{1}{R}\cdot u_R(t)$\\ \\  %3.1cm in order for the "=" to be aligned.
$u_C(t)=\frac{1}{C}\cdot \int_{0}^{t}i_C(t)dt +u_C(0)$ \hspace{1cm}  $i_C(t)=C \cdot     \frac{d u_C(t)}{dt}$

\subsection{V-I Relations to Mesh Current and Nodal Voltage}  %V-I Relations to Mesh Current and Nodal Voltage
$u_R(t)=u_A(t)-u_B(t)$  \hspace{3cm}  $i_R(t)=i(t)$ \\ \\
$u_C(t) = u_B(t)$       \hspace{4.25cm}  $i_C(t)=i(t)$  %4.25cm "=" to be aligned

\newpage
\section{Initial Conditions} 
For $t=0$:
\begin{circuitikz}[american voltages]
      \draw (0,0)
      to[V=$u_s$,invert,v=$u_s(0)$] (0,3)                % The Voltage source
      node[label={[font=\footnotesize]above:$u_A(0)$}]{} % Node A Voltage 
      to[R=$R$, v=$u_R(0)$, i=$i_R(0)$] (6,3)            % The Resistor
      node[label={[font=\footnotesize]above:$u_B(0)$}]{} % Node B Voltage 
      to[C=$C$, v=$u_C(0)$, i=$i_C(0)$] (6,0)            % The Capacitor
      to[short] (0,0);
      \draw (3,0)  node[ground]  {};                     % Ground Node
      \draw[thin, <-, >=triangle 45] (3,1.25)node{$i(0)$}  ++(-60:0.5) arc (-60:170:0.5);  %Mesh Current Arrow
\end{circuitikz}
    \\Initial Capacitor Voltage: $u_c(0)$, so from the circuit:
    \begin{align}
        u_B(0)-0& = u_c(0)  \nonumber\\
        u_B(0) &= u_c(0)    \label{eq1}
    \end{align}
 \\ Kirchoffs Current Law in node A:
\begin{align}
       u_s(0) &= u_A(0) - 0 \nonumber \\
       u_A(0) &= u_s(0)     \label{eq2}
\end{align}
Kirchoffs Current Law in node B:
\begin{align}
    i_R(0)&=i_C(0) \nonumber \\
    \frac{u_A(0)-u_B(0)}{R} &= c \frac{du_B(t)}{dt}|_{\substack{t=0}} \nonumber \\
    u_B'(0)&=\frac{1}{RC}(u_A(0)-u_B(0)) \nonumber \\
  u_B'(0)&=\frac{1}{RC}(u_s(0)-u_c(0)) \label{eq3} 
\end{align}
 \\ Kirchoffs Voltage Law in loop:
\begin{align}
      -u_s(0)+u_R(0)+u_C(0)&=0 \nonumber\\
      -u_s(0)+R*i_R(0)+u_C(0)&=0 \nonumber\\
      -u_s(0)+R*i(0)+u_C(0)&=0 \nonumber\\
      i(0)&=\frac{u_s(0)-u_C(0)}{R} \label{eqn4}
\end{align}
All initial Conditions:
\begin{align}
      u_A(0) &= u_s(0) \nonumber\\
      u_B(0) &= u_c(0)\nonumber\\
      u_B'(0)&=\frac{1}{RC}(u_s(0)-u_c(0))\nonumber\\
      i(0)&=\frac{u_s(0)-u_C(0)}{R}\nonumber
\end{align}
\section{Mesh Analysis}
\subsection{I Formula}
\begin{align*}
    -u_s(t)+u_R(t)+u_C(t)&=0 \\
    -u_s(t)+R \cdot i_R(t) + \frac{1}{C}\cdot \int_{0}^{t}i_C(t)dt +u_C(0) &=0 \\
    -u_s(t)+R \cdot i(t) + \frac{1}{C}\cdot \int_{0}^{t}i(t)dt +u_C(0) &=0\\
    \text{Laplace Transform:}     -V_s(s)+R \cdot I(s) + \frac{1}{C} \frac{I(s)}{s} +\frac{u_C(0)}{s} &=0\\
    I(s) (s+ \frac{1}{RC}) &= \frac{1}{R} (s V_s(s) - u_C(0)) \\
    I(s) = \frac{1}{R} (s V_s(&s) - u_C(0))\frac{1}{(s+ \frac{1}{RC})}
\end{align*}\\

\subsection{Solution According to Source Voltage Signal}
\subsubsection{For DC Signal}
\begin{align*}
    u_s(t) &= Us \cdot u(t)\\
    V_s(s) &= \mathcal{L} \big\{u_s(t)\big\} = \frac{U_s}{s}\\
    I(s) &= \frac{1}{R} (U_s - u_C(0))\frac{1}{(s+ \frac{1}{RC})}
\end{align*}
\subsubsection{For AC Cosine Signal}
\begin{align*}
    u_s(t) &= Us \cdot cos(\omega t)\\
    V_s(s) &= \mathcal{L} \big\{u_s(t)\big\} =U_s \frac{s}{s^2+\omega^2}\\
    I(s) &= \frac{1}{R} (U_s \frac{s}{s^2+\omega^2} - u_C(0))\frac{1}{(s+ \frac{1}{RC})}
\end{align*}
\subsection{Solution}
\subsubsection{DC Solution}
\begin{align*}
   i(t) = \frac{Us-u_C(0)}{R} e^(\frac{-t}{RC})
\end{align*}
\subsubsection{AC Cosine Solution}
\begin{align*}
   i(t) = \frac{Us\cdot C\cdot \omega \cdot sin(\omega t)-Us\cdot C^2\cdot R \cdot \omega^2 \cdot cos(\omega t)}{R^2\cdot C^2 \cdot \omega ^2 +1}-e^(\frac{-t}{RC})\cdot\frac{u_C(0)\cdot R^2\cdot C^2 \cdot \omega ^2-Us+u_C(0)}{R\cdot (R^2\cdot C^2 \cdot \omega ^2 +1)}
\end{align*}

\end{document}  
