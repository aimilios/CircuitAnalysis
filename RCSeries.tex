\documentclass{article}
\usepackage{amsmath}
\usepackage[american]{circuitikz}

\title{RC Circuit Transient Analysis using Laplace Transform for both DC and AC Voltage Source }
\author{Aimilios's Circuits}

\begin{document}

\maketitle


\begin{circuitikz}
      \draw (0,0)
      to[V,v=$u_s$] (0,3) % The voltage source
      to[R=$R$, v=$u_R$, i=$i_R$] (6,3)
      to[C=$C$, v=$u_C$, i=$i_C$] (6,0) % The capacitor
      to[short] (0,0);
\end{circuitikz}
    \\For $t=0$:
    \\Initial Capacitor Voltage: $u_c(0)$, so from the circuit:
    \begin{align}
        u_B(0)-0& = u_c(0)  \nonumber\\
        u_B(0) &= u_c(0)    \label{eq1}
    \end{align}
 \\ Kirchoffs Current Law in node A:
\begin{align}
       u_s(0) &= u_A(0) - 0 \nonumber \\
       u_A(0) &= u_s(0)     \label{eq2}
\end{align}
Kirchoffs Current Law in node B:
\begin{align}
    i_R(0)&=i_C(0) \nonumber \\
    \frac{u_A(0)-u_B(0)}{R} &= c \frac{du_B(t)}{dt}|_{\substack{t=0}} \nonumber \\
    u_B'(0)&=\frac{1}{RC}(u_A(0)-u_B(0)) \nonumber \\
  u_B'(0)&=\frac{1}{RC}(u_s(0)-u_c(0)) \label{eq3} 
\end{align}
 \\ Kirchoffs Voltage Law in loop:
\begin{align}
      -u_s(0)+u_R(0)+u_C(0)&=0 \nonumber\\
      -u_s(0)+R*i_R(0)+u_C(0)&=0 \nonumber\\
      -u_s(0)+R*i(0)+u_C(0)&=0 \nonumber\\
      i(0)&=\frac{u_s(0)-u_C(0)}{R} \label{eqn4}
\end{align}
All initial Conditions:
\begin{align}
      u_A(0) &= u_s(0) \nonumber\\
      u_B(0) &= u_c(0)\nonumber\\
      u_B'(0)&=\frac{1}{RC}(u_s(0)-u_c(0))\nonumber\\
      i(0)&=\frac{u_s(0)-u_C(0)}{R}\nonumber
\end{align}

\end{document}
